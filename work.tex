\documentclass[UTF8]{ctexart}
\newcommand{\mycmdB}[1]{{\heiti #1}}
\renewcommand{\normalsize}{\fontsize{12}{12}\fangsong}
\usepackage{listings}
\usepackage{xcolor}
\usepackage{amsmath}
\usepackage{graphicx} 
\usepackage{float} 
\usepackage{indentfirst}
\usepackage{longtable}
\usepackage{fancyhdr}
\usepackage[a4paper, left=2.5cm, right=2.5cm, top=3cm, bottom=2cm]{geometry}
\usepackage{matlab-prettifier}
\usepackage{latexcolors}

\setlength{\parindent}{2em} %2em代表首行缩进2个字符

% 页眉页脚设置
\pagestyle{fancy}
\fancyhf{}
\lhead{2021300045}
\chead{李宗霖}
\rhead{第\thepage 页}

% 去除图注冒号
\usepackage{caption}
\captionsetup[table]{labelsep=space} % 表
\captionsetup[figure]{labelsep=space} % 图 

\CTEXsetup[format={\Large\bfseries}]{section}

% 源代码引用
\renewcommand{\lstlistingname}{源代码}
\lstset{
    basicstyle          =   \zihao{5} \ttfamily,          % 基本代码风格
    keywordstyle        =   \bfseries,          % 关键字风格
    commentstyle        =   \ttfamily\itshape,  % 注释的风格,斜体
    stringstyle         =   \ttfamily,  % 字符串风格
    flexiblecolumns,                % 别问为什么,加上这个
    numbers             =   left,   % 行号的位置在左边
    showspaces          =   false,  % 是否显示空格,显示了有点乱,所以不现实了
    numberstyle         =   \zihao{5}\ttfamily,    % 行号的样式,小五号,tt等宽字体
    showstringspaces    =   false,
    captionpos          =   t,      % 这段代码的名字所呈现的位置,t指的是top上面,b指下面
    frame               =   lrtb %lrtb,   % 显示边框
}
\lstset{
	language            = matlab,
    basicstyle          = \zihao{5}\ttfamily,          % 基本代码风格
	rulesepcolor        = \color{gray}, % 代码块边框颜色
	breaklines          = true, % 代码过长则换行
	numbers             = left, % 行号在左侧显示
	numberstyle         = \zihao{5}\ttfamily,  % 行号字体
	showspaces          = false, % 不显示空格
	columns             = fixed, % 字间距固定
	%morekeywords        = {as}, % 自加新的关键字(必须前后都是空格)
	%deletendkeywords    = {compile} % 删除内定关键字;删除错误标记的关键字用deletekeywords删!
}

\lstdefinestyle{Python}{
    language        =   Python, % 语言选Python
    basicstyle      =   \zihao{5}\ttfamily,
    numberstyle     =   \zihao{5}\ttfamily,
    keywordstyle    =   \color{blue},
    keywordstyle    =   [2] \color{teal},
    stringstyle     =   \color{magenta},
    commentstyle    =   \color{red}\ttfamily,
    breaklines      =   true,   % 自动换行,建议不要写太长的行
    columns         =   fixed,  % 如果不加这一句,字间距就不固定,很丑,必须加
    basewidth       =   0.5em,
}

\begin{document}

\begin{center}
    {\zihao{-2} \bf 航天飞行动力学}\\
    {\zihao{3} 第三次作业\ ——\ 飞行方案设计}

\end{center}

\section*{\zihao{-4} 一、题目}
\noindent {\heiti 1.导弹参数:}

\begin{itemize}
    \item[*] 导弹质量$m_0=320kg$
    \item[*] 发动机推力$P=2000N$
    \item[*] 初始速度$V_0=250m/s$
    \item[*] 初始位置$x_0=0m$
    \item[*] 初始高度$H_0=7000m$
    \item[*] 初始弹道倾角$\theta=0^{\circ}$
    \item[*] 初始俯仰角 $\varphi_0=0^\circ$
    \item[*] 初始攻角 $\alpha_0=0^\circ$
    \item[*] 初始俯仰角速度$\dot{\varphi}_0=0rad$/s
    \item[*] 初始速度$V_0=250m/s$
    \item[*] 参考长度$S_{ref}=0.45 m^{2}$
    \item[*] 参考面积$L_{ref}=2.5m$
    \item[*] 升力系数$C_{y}=0.25\alpha+0.05\delta_{z}$
    \item[*] 阻力系数$C_x=0.2+0.005\alpha^2$
    \item[*] 俯仰力矩系数$m_z=-0.1\alpha+0.024\delta_z$
\end{itemize}

\noindent {\heiti 2.大气密度计算公式:}
\begin{align}
    \begin{cases}
        \rho_0=1.2495\ kg/m^3 \\
        T_0=288.15   K        \\
        T=T_0-0.0065H         \\
        \rho=\rho_{0}\left( \frac{T}{T_{0}} \right)^{4.25588}
    \end{cases}
\end{align}

\noindent {\heiti 3.飞行方案:}

\begin{itemize}
    \item[(1)] 当$x<9100m$时,采用瞬时平衡假设
        \begin{align}
            \begin{cases}
                H^*=2000\times\cos(0.000314\times1.1\times x)+5000                     \\
                \delta_z=k_\varphi\times(H\text{ -}H^*)+k_\varphi\times(H\text{ -}H^*) \\
                \delta_z=k_\varphi(H-H^*)+\dot{k}_\varphi H                            \\
                m_{s}=0.0kg/s
            \end{cases}
        \end{align}
    \item[(2)] 当$24000m>x>9100m$时,等高飞行方案,采用瞬时平衡假设。
        \begin{align}
            \begin{cases}
                H^*=3050m                                   \\
                \delta_z=k_\varphi(H-H^*)+\dot{k}_\varphi H \\
                \delta_z=k_\varphi(H-H^*)+\dot{k}_\varphi H \\
                m_s=0.46kg/s
            \end{cases}
        \end{align}
    \item[(3)] 当$x>24000m\&\&y>0$,目标位置为$x_m=30000m$,采用比例导引法和瞬时平衡假设
        \begin{align}
            \begin{cases}
                x_m=30000 m                               \\
                m_z^\alpha\alpha+m_z^{\delta_z}\delta_z=0 \\
                m_s=0.0kg/s
            \end{cases}
        \end{align}
\end{itemize}

注:舵偏角约束$\left|\delta_{z} \right|\leq 30^{\circ}$




\section*{\zihao{-4} 二、公式推导}


\noindent {\heiti 1.$x<24000m$的飞行方案:}

基于“瞬时平衡”假设,将包含20个方程的导弹运动方程组简化为铅垂平面内的质心运动方程组。

\begin{align}
    \begin{cases}
        m\frac{\mathrm{d}V}{\mathrm{d}t}=P\cos\alpha _{b}-X_{b}-mg\sin\theta\hfill                            \\
        mV\frac{\mathrm{d}\theta}{\mathrm{d}t}=P\sin\alpha _{b}+Y_{b}-mg\cos\theta\hfill                      \\
        \frac{\mathrm{d}x}{\mathrm{d}t}=V\cos\theta \hfill                                                    \\
        \frac{dy}{dt}=V\sin\theta \hfill                                                                      \\
        \frac{dm}{dt}=-m_{s} \hfill                                                                           \\
        \alpha_b=-\frac{m_{z}^{\delta_z}}{m_{z}^{\alpha}}\delta_{zb} \hfill                                   \\
        \delta_z = k_\varphi \left(H-H^{*}\right) +\dot{k}_{\varphi} \left(\dot{H} -\dot{H}^{*}\right) \hfill \\
        H^{*}=2000\times \cos\left(0.000314\times1.1\times x\right)+5000\hfill
    \end{cases}
\end{align}



\noindent {\heiti 2.$x>24000m$的飞行方案:}

\noindent {(1) 末段第一种计算方法:}


\begin{align}
    \begin{cases}
        r\frac{dq}{dt}=V_{m}\times\sin\eta-V_{T}\sin\eta_{T} \\
        \tan q=\frac{y_{T}-y_{m}}{x_{T}-x_{m}}               \\
        \frac{d\theta^*}{dt}=k\frac{dq}{dt}                  \\
        \theta^{*}-\theta_{0}=k(q-q_{0})                     \\
        \theta_{0},q_{0}?                                    \\
        \delta_{z}=k_{\theta}(\theta-\theta^{*})+k_{\dot{\theta}}(\dot{\theta}-\dot{\theta}^{*})
    \end{cases}
\end{align}

\noindent {(2) 末段第二种计算方法:}

只需要给出比例导引系数
根据运动学方程

\begin{align}
    \begin{cases}
        r\frac{dq}{dt}=V_{m}\times\sin\eta\:-V_{T}\sin\eta_{T}                                                 \\
        \begin{aligned}\tan q&=\frac{y_T-y_m}{x_T-x_m}\\
        \frac{dq}{dt}&=\frac{-V_m\sin(\theta-q)}r\end{aligned}
    \end{cases}
\end{align}

由比例导引法$\dot{\theta}^*=k\dot{q}$,可得动力学方程第二式
\begin{align}
    mV_m\dot{\theta}^*=P\sin\alpha+Y-mg\cos\theta\Rightarrow mV_mk\dot{q}=P\sin\alpha+Y-mg\cos\theta
\end{align}

由于攻角较小,进行线性化可得
\begin{align}
    mV_{m}k\dot{q}=P\alpha+Y^{\alpha}\alpha+Y^{\delta_{z}}\delta_{z}-mg\cos\theta
\end{align}
    
由于瞬时平衡$m_z=0$, 可得
\begin{align}
    -0.1\alpha+0.024\delta_{\tilde{z}}=0\Rightarrow\delta_{\tilde{z}}=0.1\alpha/0.024
\end{align}

代入,可得
\begin{align}
    \alpha=\frac{mV_{m}k\dot{q}+mg\cos\theta}{P+Y^{\alpha}+Y^{\delta_{z}}(0.1/0.024)}\Rightarrow\frac{mV_{m}k\dot{q}+mg\cos\theta}{P+C_{y}^{\alpha}qL_{ref}+C_{y}^{\delta_{z}}qL_{ref}(0.1/0.024)}
\end{align}
    

最后得到弹道方程为

\begin{align}
    \begin{cases}
        \frac{dV}{dt}=\frac{P\cos\alpha-X}{m}-g\sin\theta  \\
        \alpha=\frac{mVk\dot{q}+mg\cos\theta}{P+C_{y}^{\alpha}qL_{ref}+C_{y}^{\delta_{z}}qL_{ref}(0.1/0.024)} \\
        \frac{dx}{dt}=V\cos\theta  \\
        {\frac{dy}{dt}}=V\sin\theta  \\
        \dot{\theta}^{*}=k\dot{q} \\
        \dot{\theta}^{*}=\dot{\theta} \\
        \tan q=\frac{y_{T}-y_{m}}{x_{T}-x_{m}} \\
        \frac{dq}{dt}=\frac{-V\sin(\theta-q)}{r} \\
        \delta_{z}=0.1\alpha/0.024
    \end{cases}
\end{align}

推力{\bf{P}}在弹体坐标系$Ox_{1}y_{1}z_{2}$中的分量用$P_{x1}$、$P_{y1}$、$P_{z1}$表示,则根据题目信息可以得到
\begin{align}
    \begin{cases}
        \frac{P_{y1}}{P_{x1}}=\tan{\left( - 30^\circ\right)} \\
        \frac{P_{y1}}{P_{z1}}=\tan{45^\circ}
        \\
        P_{x1}^2 + P_{y1}^2 + P_{z1}^2 = P
    \end{cases}
\end{align}

带入$\vec{P}=50000N$求得推力大小:
\begin{align}
    \begin{pmatrix}
        P_{x1} \\
        P_{y1} \\
        P_{z1}
    \end{pmatrix}
    =
    \begin{pmatrix}
        38730  \\
        -22361 \\
        -22361
    \end{pmatrix}
\end{align}

\noindent {\bf (1) 按照$\psi \rightarrow \varphi \rightarrow \gamma$的顺序转换}

从地面坐标系$Axyz$转换为弹体坐标系$Ox_{1}y_{1}z_{2}$的坐标变换矩阵为
\begin{align}
    L \left(\psi , \varphi , \gamma\right)
    =
    L_{x} \left(\gamma\right)
    L_{z} \left(\varphi\right)
    L_{y} \left(\psi\right)
\end{align}
其中:
\begin{align}
    L_{y} \left(\psi\right)
    =
    \begin{pmatrix}
        cos\psi & 0 & -sin\psi
        \\
        0       & 1 & 0
        \\
        sin\psi & 0 & cos\psi
    \end{pmatrix}
\end{align}

\begin{align}
    L_{z} \left(\varphi \right)
    =
    \begin{pmatrix}
        cos\varphi  & sin\varphi & 0
        \\
        -sin\varphi & cos\varphi & 0
        \\
        0           & 0          & 1
    \end{pmatrix}
\end{align}

\begin{align}
    L_{x} \left(\gamma\right)
    =
    \begin{pmatrix}
        1 & 0          & 0
        \\
        0 & cos\gamma  & sin\gamma
        \\
        0 & -sin\gamma & cos\gamma
    \end{pmatrix}
\end{align}

代入则有:
\begin{align}
    L \left(\psi , \varphi , \gamma\right)
    =
    \begin{pmatrix}
        cos\varphi cos\psi                                & \sin\varphi           & -cos\varphi\sin\psi
        \\
        -\sin\varphi cos\psi cos\gamma+\sin\psi\sin\gamma & cos\varphi\cos\gamma  & \sin\varphi\sin\psi\cos\gamma+cos\psi\sin\gamma
        \\
        \sin\varphi\cos\psi\sin\gamma+\sin\psi\cos\gamma  & -cos\varphi\sin\gamma & -\sin\varphi\sin\psi\sin\gamma+cos\psi\cos\gamma
    \end{pmatrix}
\end{align}

同时,$L^{-1} \left(\psi , \varphi , \gamma\right) = L^{T} \left(\psi , \varphi , \gamma\right)$,所以

\begin{align}
    L^{T} \left(\psi , \varphi , \gamma\right)
    =
    \begin{pmatrix}
        \cos\varphi\cos\psi  & -\sin\varphi\cos\psi\cos\gamma+\sin\psi\sin\gamma & \sin\varphi\cos\psi\sin\gamma+\sin\psi\cos\gamma
        \\
        \sin\varphi          & \cos\varphi\cos\gamma                             & -\cos\varphi\sin\gamma
        \\
        -\cos\varphi\sin\psi & \sin\varphi\sin\psi\cos\gamma+\cos\psi\sin\gamma  & -\sin\varphi\sin\psi\sin\gamma+\cos\psi\cos\gamma
    \end{pmatrix}
\end{align}

发动机推力矢量$\vec{P}$在地面坐标系$Axyz$的投影为

\begin{align}
    \begin{pmatrix}
        P_{x} \\
        P_{y} \\
        P_{z}
    \end{pmatrix}
    =
    L^{T} \left(\psi , \varphi , \gamma\right)
    \begin{pmatrix}
        P_{x1} \\
        P_{y1} \\
        P_{z1}
    \end{pmatrix}
\end{align}

\noindent {\bf (2) 按照$\varphi \rightarrow \psi \rightarrow \gamma$的顺序转换}

从地面坐标系$Axyz$转换为弹体坐标系$Ox_{1}y_{1}z_{2}$的坐标变换矩阵为
\begin{align}
    L \left(\varphi , \psi , \gamma\right)
    =
    L_{x} \left(\gamma\right)
    L_{y} \left(\psi\right)
    L_{z} \left(\varphi\right)
\end{align}

代入式$\left(4\right)$、式$\left(5\right)$、式$\left(6\right)$得到:
\begin{align}
    L \left(\varphi , \psi , \gamma\right)
    =
    \begin{pmatrix}
        \cos\varphi\cos\psi                                 & \sin\varphi\cos\psi                                 & -\sin\varphi
        \\
        \sin\psi\cos\varphi\sin\gamma-\sin\varphi\cos\gamma & \sin\psi\sin\varphi\sin\gamma+\cos\varphi\cos\gamma & \cos\psi\sin\gamma
        \\
        \sin\psi\cos\varphi\cos\gamma+\sin\varphi\cos\gamma & \sin\psi\sin\varphi\cos\gamma-\cos\varphi\sin\gamma & \cos\psi\cos\gamma
    \end{pmatrix}
\end{align}

同时,$L^{-1} \left(\psi , \varphi , \gamma\right) = L^{T} \left(\psi , \varphi , \gamma\right)$,所以

\begin{align}
    L^{T} \left(\psi , \varphi , \gamma\right)
    =
    \begin{pmatrix}
        \cos\varphi\cos\psi  & -\sin\varphi\cos\psi\cos\gamma+\sin\psi\sin\gamma & \sin\varphi\cos\psi\sin\gamma+\sin\psi\cos\gamma
        \\
        \sin\varphi          & \cos\varphi\cos\gamma                             & -\cos\varphi\sin\gamma
        \\
        -\cos\varphi\sin\psi & \sin\varphi\sin\psi\cos\gamma+\cos\psi\sin\gamma  & -\sin\varphi\sin\psi\sin\gamma+\cos\psi\cos\gamma
    \end{pmatrix}
\end{align}

发动机推力矢量$\vec{P}$在地面坐标系$Axyz$的投影为

\begin{align}
    \begin{pmatrix}
        P_{x} \\
        P_{y} \\
        P_{z}
    \end{pmatrix}
    =
    L^{T} \left(\psi , \varphi , \gamma\right)
    \begin{pmatrix}
        P_{x1} \\
        P_{y1} \\
        P_{z1}
    \end{pmatrix}
\end{align}

\section*{\zihao{-4} 二、编程计算结果}

\noindent {\bf (1) 按照$\psi \rightarrow \varphi \rightarrow \gamma$的顺序转换计算结果}
\begin{align}
    \begin{pmatrix}
        P_{x} \\
        P_{y} \\
        P_{z}
    \end{pmatrix}
    =
    \begin{pmatrix}
        2996   \\
        -49152 \\
        8667
    \end{pmatrix}
    \left(N\right)
\end{align}
\noindent {\bf (2) 按照$\varphi \rightarrow \psi \rightarrow \gamma$的顺序转换}
\begin{align}
    \begin{pmatrix}
        P_{x} \\
        P_{y} \\
        P_{z}
    \end{pmatrix}
    =
    L^{T} \left(\psi , \varphi , \gamma\right)
    \begin{pmatrix}
        11180  \\
        -47405 \\
        -11305
    \end{pmatrix}
    \left(N\right)
\end{align}


\clearpage
\lstinputlisting[
    style       =   Python,
    caption     =   {\bf main.py},
    label       =   {main.py}
]{./code/main.py}

\clearpage

\lstinputlisting[
    style       =   Matlab-editor,
    caption     =   {\bf work.tex},
    label       =   {work.tex}
]{work.tex}

\end{document}
